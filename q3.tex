\documentclass{beamer}


\mode<presentation>
{
  \usetheme{Warsaw}
  

  \setbeamercovered{transparent}
  % or whatever (possibly just delete it)
}


\usepackage[english]{babel}
\usepackage{graphicx}

\usepackage[utf8]{inputenc}
% or whatever

\usepackage{times}
\usepackage[T1]{fontenc}
% Or whatever. Note that the encoding and the font should match. If T1
% does not look nice, try deleting the line with the fontenc.


\title[Amoeba OS] % (optional, use only with long paper titles)
{Amoeba Distributed Operating System}


\author[Lyness Hill] % (optional, use only with lots of authors)
{Lyness Hill}
% - Give the names in the same order as the appear in the paper.
% - Use the \inst{?} command only if the authors have different
%   affiliation.

\institute[Loyola University Chicago] % (optional, but mostly needed)
{
  Lyness Hill
  Department of Computer Science\\
  Loyola University Chicago 
}

% - Use the \inst command only if there are several affiliations.
% - Keep it simple, no one is interested in your street address.

\date[12/14/11] % (optional, should be abbreviation of conference name)
% - Either use conference name or its abbreviation.
% - Not really informative to the audience, more for people (including
%   yourself) who are reading the slides online

\subject{Operating Systems}
% This is only inserted into the PDF information catalog. Can be left
% out. 



% If you have a file called "university-logo-filename.xxx", where xxx
% is a graphic format that can be processed by latex or pdflatex,
% resp., then you can add a logo as follows:

% \pgfdeclareimage[height=0.5cm]{university-logo}{university-logo-filename}
% \logo{\pgfuseimage{university-logo}}


\begin{document}


\begin{frame}
  \titlepage
  \end{frame}

\begin{frame}
  \includegraphics[scale=0.5] {Amoeba}
\end{frame}

\section{Introduction}

\subsection{History}

\begin{frame}{History}
 	\begin{itemize}
	\item Created by Virije University Prof. Andrew S. Tananbaum and Sape Mullender and Robert van Renesse (two of Tananbaum's students)
	\item Began development in 1981
	\item First release in 1983
	\item Last update was 1996 
	\end{itemize} 
\end{frame}

\subsection{The Concept}

\begin{frame}{The Concept}
	""Provide users with the illusion of a single, powerful, timesharing system" 
\end{frame}

\subsection{Design Goals}

\begin{frame}{Design Goals}
	\begin{itemize}
	\item Small Distributed Operating System
	\item Fault Tolerance
	\item Parallelism 
	\item Transparency
	\end{itemize}
\end{frame}

\subsection{Architecture}

\begin{frame}{Architecture}
\includegraphics[scale=0.5]{Fig21}
\end{frame}

\begin{frame}{Architecture}
	\begin{itemize}
	\item Processor Pool
		\begin{itemize}
		\item Multiple CPU's
		\item Assigns processors as needed 
		\end{itemize}
	\item Workstations (Disk-less)
	\item WAN Gateway
	\item Specialized Server
	\end{itemize}

\end{frame}

\section{The System}

\subsection{Processes And Threads}

\begin{frame}{Processes And Threads}
\end{frame}

\subsection{Scheduling}
\begin{frame}{Scheduling}
\end{frame}

\subsection{Kernel Memory Management}
\begin{frame}{Kernel Memory Management}
\end{frame}

\section{What Makes A Good Operating System?}

\subsection {What Makes A Good Operating System}
\begin{frame}{Make Titles Informative. Use Uppercase Letters.}{Subtitles are optional.}
  % - A title should summarize the slide in an understandable fashion
  %   for anyone how does not follow everything on the slide itself.

  \begin{itemize}
  \item
    Use \texttt{itemize} a lot.
  \item
    Use very short sentences or short phrases.
  \end{itemize}
\end{frame}

\begin{frame}{Make Titles Informative.}

  You can create overlays\dots
  \begin{itemize}
  \item using the \texttt{pause} command:
    \begin{itemize}
    \item
      First item.
      \pause
    \item    
      Second item.
    \end{itemize}
  \item
    using overlay specifications:
    \begin{itemize}
    \item<3->
      First item.
    \item<4->
      Second item.
    \end{itemize}
  \item
    using the general \texttt{uncover} command:
    \begin{itemize}
      \uncover<5->{\item
        First item.}
      \uncover<6->{\item
        Second item.}
    \end{itemize}
  \end{itemize}
\end{frame}


\subsection{Previous Work}

\begin{frame}{Make Titles Informative.}
\end{frame}

\begin{frame}{Make Titles Informative.}
\end{frame}



\section{Our Results/Contribution}

\subsection{Main Results}

\begin{frame}{Make Titles Informative.}
\end{frame}

\begin{frame}{Make Titles Informative.}
\end{frame}

\begin{frame}{Make Titles Informative.}
\end{frame}


\subsection{Basic Ideas for Proofs/Implementation}

\begin{frame}{Make Titles Informative.}
\end{frame}

\begin{frame}{Make Titles Informative.}
\end{frame}

\begin{frame}{Make Titles Informative.}
\end{frame}



\section{Summary}

\begin{frame}{Summary}

  % Keep the summary *very short*.
  \begin{itemize}
  \item
    The \alert{first main message} of your talk in one or two lines.
  \item
    The \alert{second main message} of your talk in one or two lines.
  \item
    Perhaps a \alert{third message}, but not more than that.
  \end{itemize}
  
  % The following outlook is optional.
  \vskip0pt plus.5fill
  \begin{itemize}
  \item
    Outlook
    \begin{itemize}
    \item
      Something you haven't solved.
    \item
      Something else you haven't solved.
    \end{itemize}
  \end{itemize}
\end{frame}



% All of the following is optional and typically not needed. 
\appendix
\section<presentation>*{\appendixname}
\subsection<presentation>*{For Further Reading}

\begin{frame}[allowframebreaks]
  \frametitle<presentation>{For Further Reading}
    
  \begin{thebibliography}{10}
    
  \beamertemplatebookbibitems
  % Start with overview books.

  \bibitem{Author1990}
    A.~Author.
    \newblock {\em Handbook of Everything}.
    \newblock Some Press, 1990.
 
    
  \beamertemplatearticlebibitems
  % Followed by interesting articles. Keep the list short. 

  \bibitem{Someone2000}
    S.~Someone.
    \newblock On this and that.
    \newblock {\em Journal of This and That}, 2(1):50--100,
    2000.
  \end{thebibliography}
\end{frame}

\end{document}
